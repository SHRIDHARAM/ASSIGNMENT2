\documentclass[journal,12pt,twocolumn]{IEEEtran}
\usepackage[margin=1in,top=0.7cm, footskip=0.25in]{geometry}
\usepackage{amsmath}
\usepackage{amsthm}
\usepackage{mathtools}
\usepackage{textcomp}
\usepackage{listings}



\title{\textbf{\Huge Assignment 2}}
\author{\textbf{\Large Shridharam Tiwari - SM21MTECH12003}}


\begin{document}
 



\providecommand{\mbf}{\mathbf}
\providecommand{\norm}[1]{\left\lVert#1\right\rVert}
\newcommand{\myvec}[1]{\ensuremath{\begin{pmatrix}#1\end{pmatrix}}}
\let\vec\mathbf

\maketitle



\section*{Ch-3, Ex-5, Q.3}
\vspace{0.35cm}



\textbf{1. Find tbe bisectors of acute angles between tbe pairs of
lines\\
\vspace{0.1cm} }

\begin{align}
3x+4y-9=0  \\ 12x+5y-3=0
\end{align}


\vspace{0.1cm}

\textbf{Solution:}  

\vspace{0.2cm}
Angle bisector of two lines is the line which bisects the angle between the two lines is the locus of a point which is equidistant from the two lines. Suppose we have two lines 

\begin{align}
L_1 : A_1x + B_1y + C_1 = 0 \\
L_2 : A_2x + B_2y + C_2 = 0 
\end{align}


\vspace{0.2cm} 

The angle bisectors is given by,
\begin{align}
\frac{A_1x + B_1y + C_1}{\sqrt{(A_1)^2 + (B_1)^2}} = \pm \frac{A_2x + B_2y + C_2}{\sqrt{(A_2)^2 + (B_2)^2}}
\end{align}


\vspace{0.5cm} 
\textbf{(i) Check where origin lies}
\vspace{0.5cm} 

We need to check the sign of {C_1C_2(A_1A_2+B_1B_2).} \\

The positive sign refers that the origin is contained by the obtuse angle, and the negative sign refers that origin is contained by the acute angle. 
From our equations we are getting positive sign which means that origin is contained by obtuse angle. Acute angle doesn't contain origin.  

Since the sign of constants of both the lines are same, the angle bisector not containing origin is given by,
\begin{align}
\frac{A_1x + B_1y + C_1}{\sqrt{(A_1)^2 + (B_1)^2}} = - \frac{A_2x + B_2y + C_2}{\sqrt{(A_2)^2 + (B_2)^2}}
\end{align}

\begin{align}
\frac{3x + 4y - 9}{\sqrt{(3)^2 + (4)^2}} = - \frac{12x + 5y - 3}{\sqrt{(12)^2 + (5)^2}}
\end{align}

\begin{align}
\frac{3x + 4y - 9}{5} = - \frac{12x + 5y - 3}{13}
\end{align}

By solving above equation, the bisector which bisects the angle not containing the origin i.e acute angle bisector is given by,

\begin{align}
9x + 7y - 12 = 0 
\end{align}


The other bisector is given by,
\begin{align}
7x - 9y + 34 = 0
\end{align}



\textbf{Download python code at }
\begin{lstlisting}[frame=single] 
https://github.com/SHRIDHARAM
/ASSIGNMENT2
\end{lstlisting}



\end{document}

